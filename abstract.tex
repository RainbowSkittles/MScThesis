\begin{abstract}
    Queueing network theory is an important probabilistic analytical modelling tool for computer performance analysis, when designing and optimizing computer systems. A number of exact methods exist for the calculation of performance measures and state probability normalizing constants. The latter quantity is important as it appears in likelihood functions for the inference of queueing network models based on measurement data. These exact algorithms include the Convolution algorithm, RECAL, and MoM algorithms. While these provide exact answers, they generally do not scale well with the number of stations, customer classes and customer populations in the model.
    \\\\
    A new integral form of the normalizing constant, which takes the form of an integral on the simplex is proposed. Through the logistic (or log-ratio) transform, Laplace's method could be applied to compute this quantity. Laplace's method's applicability to this problem indicates the suitability of Importance Sampling on the integration domain, centered around the integrand's stationary point.
    \\\\
    This project focuses on the implementation of the Logistic Sampling algorithm as it was initially proposed, in Java. This involved a thorough understanding of the proofs and methods that make this computation possible. Novel contributions to this project include several new results concerning different transform types. Besides that, this project was able to see the derivation of results for, and implementation of an extended Logistic Sampling algorithm that catered for multi-server models.
\end{abstract}